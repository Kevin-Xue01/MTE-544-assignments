\section{Differential-Drive Simulation [20 pts]}
Create a simulation for the 2WD robot with \underline{Python} which takes as inputs the linear and angular velocities $\begin{bmatrix}
    v,\omega
\end{bmatrix}^\top$ (you can use the state space form). You should obtain the poses of the robot $\begin{bmatrix}
    x(t),y(t),\theta(t)
\end{bmatrix}^\top$ as outputs. 
\begin{enumerate}[(a)]
    \item Simulate and plot the trajectories obtained for the following velocities (use at least 30 seconds with a timestep of 0.1s):
    \begin{enumerate}[i.]
        \item Constant velocities
        \begin{enumerate} [1.]
            \item $\begin{bmatrix}
    v,\omega
\end{bmatrix}^\top=\begin{bmatrix}
    1,0
\end{bmatrix}^\top$
\item $\begin{bmatrix}
    v,\omega
\end{bmatrix}^\top=\begin{bmatrix}
    0,0.3
\end{bmatrix}^\top$
\item $\begin{bmatrix}
    v,\omega
\end{bmatrix}^\top=\begin{bmatrix}
    1,0.3
\end{bmatrix}^\top$
        \end{enumerate}
        \item Velocity profiles: $\begin{bmatrix}
            v(t)\\\omega(t)
        \end{bmatrix}=\begin{bmatrix}
            1+0.1\cdot\sin(t)\\
            0.2+0.5\cdot\cos(t)
        \end{bmatrix}$
    \end{enumerate}
    Linear velocities are [m/s] and angular velocities are [rad/s]. For each case, plot the top view of the 2D trajectory ($x$ vs $y$), and then each variable against time, including orientation $\theta$. Discuss briefly your observations on the outcome (feel free to perform more tests). [15 pts]
    \item 	Assuming the robot has dimensions $T$ (track) $= 0.2$ [m], and wheel radius $r = 0.1$ [m], what are the wheel speeds $u_l$,$u_r$ needed to achieve the constant velocities in (a).i? 
    
    Briefly comment on how you would need to change your implemented simulation if instead of linear and angular velocities, the input control parameters were wheel speeds.


\end{enumerate}