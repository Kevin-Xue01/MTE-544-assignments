\documentclass{article}
\usepackage{graphicx} % Required for inserting images
\usepackage{amsmath}
\usepackage{bm} % For bold math symbols

\title{MTE 544 Assignment 1}
\author{Kevin Xue, Student ID: 20814292}

\begin{document}
\maketitle

\section{}
\[
\text{Rotation matrix: }
\bm{\mathbf{R}} = \begin{bmatrix}
\cos(\theta) & -\sin(\theta) \\
\sin(\theta) & \cos(\theta)
\end{bmatrix},
\quad
\text{Translation vector: }
\bm{\mathbf{q}} = \begin{bmatrix}
q_x \\
q_y
\end{bmatrix}
\]
\[
\bm{\mathbf{G}} = \begin{bmatrix}
\cos(\theta) & -\sin(\theta) & q_x \\
\sin(\theta) & \cos(\theta)  & q_y \\
0            & 0             & 1
\end{bmatrix}
\]
\rule{\linewidth}{0.2mm}

\begin{equation}
\bm{\mathbf{R^T}} = \begin{bmatrix}
\cos(\theta) & \sin(\theta) \\
-\sin(\theta) & \cos(\theta)
\end{bmatrix}
\end{equation}
\quad
\begin{equation}
\bm{\mathbf{0}_{1 \times 2}} = \begin{bmatrix}
0 & 0
\end{bmatrix}
\end{equation}

\begin{equation}
-\bm{\mathbf{R^T}} \cdot \bm{\mathbf{q}} = -\begin{bmatrix}
\cos(\theta) & \sin(\theta) \\ -\sin(\theta) & \cos(\theta)
\end{bmatrix} \begin{bmatrix}
q_x \\ q_y
\end{bmatrix}
= \begin{bmatrix}
- \sin(\theta)q_y - \cos(\theta)q_x  \\ -\cos(\theta)q_y + \sin(\theta)q_x 
\end{bmatrix}
\end{equation}
\rule{\linewidth}{0.2mm}
\[
\bm{\mathbf{G^{-1}}} = \frac{\text{adj}(\bm{\mathbf{G}})}{\det(\bm{\mathbf{G}})}
\]

The determinant of \( \bm{\mathbf{G}} \) can be evaluated using cofactor expansion along the last row:
\[
\det(\bm{\mathbf{G}}) = 0 \cdot \det \begin{bmatrix} -\sin(\theta) & q_x \\ \cos(\theta) & q_y \end{bmatrix} 
- 0 \cdot \det \begin{bmatrix} \cos(\theta) & q_x \\ \sin(\theta) & q_y \end{bmatrix} 
+ 1 \cdot \det \begin{bmatrix} \cos(\theta) & -\sin(\theta) \\ \sin(\theta) & \cos(\theta) \end{bmatrix}
\]

Since the first two terms are multiplied by 0, the determinant simplifies to:
\[
\det(\bm{\mathbf{G}}) = \det \begin{bmatrix} \cos(\theta) & -\sin(\theta) \\ \sin(\theta) & \cos(\theta) \end{bmatrix} = 1
\]
Thus:
\[
\bm{\mathbf{G^{-1}}} = \text{adj}(\bm{\mathbf{G}})
\]

Adjoint of \( \bm{\mathbf{G}} \)
\[
\text{adj}(\bm{\mathbf{G}}) = \begin{bmatrix}
+ \det\begin{bmatrix} \cos(\theta) & q_y \\ 0 & 1 \end{bmatrix} & - \det\begin{bmatrix} -\sin(\theta) & q_x \\ 0 & 1 \end{bmatrix} & + \det\begin{bmatrix} -\sin(\theta) & q_x \\ \cos(\theta) & q_y \end{bmatrix} \\
- \det\begin{bmatrix} \sin(\theta) & q_y \\ 0 & 1 \end{bmatrix} & + \det\begin{bmatrix} \cos(\theta) & q_x \\ 0 & 1 \end{bmatrix} & - \det\begin{bmatrix} \cos(\theta) & q_x \\ \sin(\theta) & q_y \end{bmatrix} \\
+ \det\begin{bmatrix} \sin(\theta) & \cos(\theta) \\ 0 & 0 \end{bmatrix} & - \det\begin{bmatrix} \cos(\theta) & -\sin(\theta) \\ 0 & 0 \end{bmatrix} & + \det\begin{bmatrix} \cos(\theta) & -\sin(\theta) \\ \sin(\theta) & \cos(\theta) \end{bmatrix}
\end{bmatrix}
\]

\[
\text{adj}(\bm{\mathbf{G}}) = \begin{bmatrix}
\cos(\theta) & \sin(\theta) & -\sin(\theta)q_y - \cos(\theta)q_x \\
-\sin(\theta) & \cos(\theta) & -\cos(\theta)q_y + \sin(\theta)q_x \\
0 & 0 & 1
\end{bmatrix}
\]

Substituting (1), (2), and (3) into the adjoint of \( \bm{\mathbf{G}} \)

\[
\bm{\mathbf{G^{-1}}} = \text{adj}(\bm{\mathbf{G}}) = \begin{bmatrix}
\cos(\theta) & \sin(\theta) & -\sin(\theta)q_y - \cos(\theta)q_x \\
-\sin(\theta) & \cos(\theta) & -\cos(\theta)q_y + \sin(\theta)q_x \\
0 & 0 & 1
\end{bmatrix}
= \begin{bmatrix}\bm{\mathbf{R^T}} & -\bm{\mathbf{R^T}} \cdot \bm{\mathbf{q}} \\
\bm{\mathbf{{0}_{1 \times 2}}} & 1
\end{bmatrix}
\]

% \bm{G}^{s}_{e} = \bm{G}^{s}_{1} \cdot \bm{G}^{1}_{2} \cdot \bm{G}^{2}_{3} \cdot \bm{G}^{3}_{e}
\newpage
\section{}
(a)
\[
\bm{\mathbf{G^{s}_{e}}} = \bm{\mathbf{G^{s}_{1}}} \cdot \bm{\mathbf{G^{1}_{2}}} \cdot \bm{\mathbf{G^{2}_{3}}} \cdot \bm{\mathbf{G^{3}_{e}}}
\]

\[
\bm{\mathbf{G_{e}^{3}}} =
\begin{bmatrix}
\cos(-\frac{\pi}{2}) & -\sin(-\frac{\pi}{2}) & l_3 \\
\sin(-\frac{\pi}{2}) & \cos(-\frac{\pi}{2})  & 0 \\
0 & 0 & 1
\end{bmatrix} =
\begin{bmatrix}
0 & 1 & l_3 \\
-1 & 0 & 0 \\
0 & 0 & 1
\end{bmatrix}
\]

\[
\bm{\mathbf{G_{3}^{2}}} =
\begin{bmatrix}
\cos(\theta) & -\sin(\theta) & l_2 \\
\sin(\theta) & \cos(\theta)  & 0 \\
0 & 0 & 1
\end{bmatrix}
\]

\[
\bm{\mathbf{G_{2}^{1}}} =
\begin{bmatrix}
\cos(\gamma) & -\sin(\gamma) & l_1 \\
\sin(\gamma) & \cos(\gamma)  & 0 \\
0 & 0 & 1
\end{bmatrix}
\]

\[
\bm{\mathbf{G_{1}^{s}}} =
\begin{bmatrix}
\cos(\alpha) & -\sin(\alpha) & x \\
\sin(\alpha) & \cos(\alpha)  & y \\
0 & 0 & 1
\end{bmatrix}
\]

% Detailed Evaluation
To evaluate \(\bm{\mathbf{G^{s}_{e}}}\), we compute the product of the matrices from right to left:

\[
\bm{\mathbf{G^{s}_{e}}} = 
\begin{bmatrix}
\cos(\alpha) & -\sin(\alpha) & x \\
\sin(\alpha) & \cos(\alpha)  & y \\
0 & 0 & 1
\end{bmatrix} \cdot
\begin{bmatrix}
\cos(\gamma) & -\sin(\gamma) & l_1 \\

\sin(\gamma) & \cos(\gamma)  & 0 \\
0 & 0 & 1
\end{bmatrix} \cdot
\begin{bmatrix}
\cos(\theta) & -\sin(\theta) & l_2 \\
\sin(\theta) & \cos(\theta)  & 0 \\
0 & 0 & 1
\end{bmatrix} \cdot
\begin{bmatrix}
0 & 1 & l_3 \\
-1 & 0 & 0 \\
0 & 0 & 1
\end{bmatrix} 
\]

\[
\bm{\mathbf{G^{s}_{e}}} = 
\begin{bmatrix}
\cos(\alpha) & -\sin(\alpha) & x \\
\sin(\alpha) & \cos(\alpha)  & y \\
0 & 0 & 1
\end{bmatrix} \cdot
\begin{bmatrix}
\cos(\gamma) & -\sin(\gamma) & l_1 \\

\sin(\gamma) & \cos(\gamma)  & 0 \\
0 & 0 & 1
\end{bmatrix} \cdot
\begin{bmatrix}
\sin(\theta) & \cos(\theta) & l_3 \cos(\theta) + l_2 \\
-\cos(\theta) & \sin(\theta) & l_3 \sin(\theta) \\
0 & 0 & 1
\end{bmatrix}
\]

\[
\bm{\mathbf{G^{s}_{e}}} = 
\begin{bmatrix}
\cos(\alpha) & -\sin(\alpha) & x \\
\sin(\alpha) & \cos(\alpha)  & y \\
0 & 0 & 1
\end{bmatrix} \cdot
\begin{bmatrix}
\sin(\gamma + \theta) & \cos(\gamma + \theta) & l_1 + l_2 \cos(\gamma) + l_3 \cos(\gamma + \theta) \\
-\cos(\gamma + \theta) & \sin(\gamma + \theta) & l_2 \sin(\gamma) + l_3 \sin(\gamma + \theta) \\
0 & 0 & 1
\end{bmatrix}
\]

\[
\bm{\mathbf{G^{s}_{e}}} = 
\begin{bmatrix}
\sin(\alpha + \gamma + \theta) & \cos(\alpha + \gamma + \theta) & l_1\cos(\alpha) + l_2 \cos(\alpha + \gamma) + l_3 \cos(\alpha + \gamma + \theta) + x \\
-\cos(\alpha + \gamma + \theta) & \sin(\alpha + \gamma + \theta) & l_1\sin(\alpha) + l_2 \sin(\alpha + \gamma) + l_3 \sin(\alpha + \gamma + \theta) + y \\
0 & 0 & 1
\end{bmatrix}
\]
(b)
\[
\langle \alpha, \gamma, \theta \rangle = \left\langle \frac{\pi}{4}, \frac{\pi}{4}, -\frac{\pi}{3} \right\rangle, \quad l_1 = l_2 = l_3 = 1, \quad x = y = 1
\]
Substituting the above values into the result of 2(a):
\[
\bm{\mathbf{G^{s}_{e}}} = 
\begin{bmatrix}
\frac{1}{2} & \frac{\sqrt{3}}{2} & \frac{\sqrt{2} + \sqrt{3}}{2} + 1 \\
-\frac{\sqrt{3}}{2} & \frac{1}{2} & \frac{5+\sqrt{2}}{2} \\
0 & 0 & 1
\end{bmatrix}
\]
Homogenous coordinate representations:

\[
\bm{\mathbf{\overline{p}^{e}_{1}}}
= \begin{bmatrix}
    \bm{\mathbf{p^e_1}} \\ 1
\end{bmatrix}
= \begin{bmatrix}
    0 \\ 0 \\ 1
\end{bmatrix},
\bm{\mathbf{\overline{p}^{e}_{2}}}
= \begin{bmatrix}
    \bm{\mathbf{p^e_2}} \\ 1
\end{bmatrix}
= \begin{bmatrix}
    1 \\ 2 \\ 1
\end{bmatrix}
\]

Solve for: \(\bm{\mathbf{p^{s}_{1}}}\):
\[
\bm{\mathbf{\overline{p}^{s}_{1}}} = \bm{\mathbf{G^{s}_{e}}} \cdot \bm{\mathbf{\overline{p}^{e}_{1}}} = \begin{bmatrix}
\frac{1}{2} & \frac{\sqrt{3}}{2} & \frac{\sqrt{2} + \sqrt{3}}{2} + 1 \\
-\frac{\sqrt{3}}{2} & \frac{1}{2} & \frac{5+\sqrt{2}}{2} \\
0 & 0 & 1
\end{bmatrix} \cdot \begin{bmatrix} 0 \\ 0 \\ 1 \end{bmatrix} = 
\begin{bmatrix}
    \frac{\sqrt{2} + \sqrt{3}}{2} + 1 \\ \frac{5+\sqrt{2}}{2} \\ 1
\end{bmatrix}\]
\[
\bm{\mathbf{p^{s}_{1}}} = \begin{bmatrix}
    2.573132185 \\ 3.207106781
\end{bmatrix}
\]
Solve for: \(\bm{\mathbf{p^{s}_{2}}}\):
\[
\bm{\mathbf{\overline{p}^{s}_{2}}} = \bm{\mathbf{G^{s}_{e}}} \cdot \bm{\mathbf{\overline{p}^{e}_{2}}} = \begin{bmatrix}
\frac{1}{2} & \frac{\sqrt{3}}{2} & \frac{\sqrt{2} + \sqrt{3}}{2} + 1 \\
-\frac{\sqrt{3}}{2} & \frac{1}{2} & \frac{5+\sqrt{2}}{2} \\
0 & 0 & 1
\end{bmatrix} \cdot \begin{bmatrix} 1 \\ 2 \\ 1 \end{bmatrix} = 
\begin{bmatrix}
    \frac{1}{2} + \sqrt{3} + \frac{\sqrt{2} + \sqrt{3}}{2} + 1 \\ 
    -\frac{\sqrt{3}}{2} + 1 + \frac{5+\sqrt{2}}{2} \\ 1
\end{bmatrix} 
\]
\[
\bm{\mathbf{p^{s}_{2}}} = \begin{bmatrix}
    4.805182993 \\ 3.341081377
\end{bmatrix}
\]
(c)

\[
\bm{\mathbf{\overline{p}^s}} = \bm{\mathbf{G^{s}_{2}}} \cdot \bm{\mathbf{\overline{p}^2}}
\]
Given the fact that point \(\bm{\mathbf{p}}\) is rigidly fixed to frame 2, the following holds true:
\[
\bm{\mathbf{\dot{\overline{p}}^s}} = \bm{\mathbf{\dot{G^{s}_{2}}}} \cdot \bm{\mathbf{\overline{p}^2}}
\]
Where
\[
\bm{\mathbf{G^{s}_{2}}} = \bm{\mathbf{G^{s}_{1}}} \cdot \bm{\mathbf{G^{1}_{2}}} = \begin{bmatrix}
\cos(\alpha) & -\sin(\alpha) & x \\
\sin(\alpha) & \cos(\alpha)  & y \\
0 & 0 & 1
\end{bmatrix} \cdot
\begin{bmatrix}
\cos(\gamma) & -\sin(\gamma) & l_1 \\

\sin(\gamma) & \cos(\gamma)  & 0 \\
0 & 0 & 1
\end{bmatrix}
\]
Substituting \(x = 0, y = 0, l_1 = 1\) and multiplying through:
\[
\bm{\mathbf{G^{s}_{2}}} =
\begin{bmatrix}
\cos(\alpha + \gamma) & -\sin(\alpha + \gamma) & \cos(\alpha) \\
\sin(\alpha + \gamma) & \cos(\alpha + \gamma) & \sin(\alpha) \\
0 & 0 & 1
\end{bmatrix}
\]
Thus:
\[
\bm{\mathbf{\dot{G^{s}_{2}}}} = \begin{bmatrix}
    -\sin(\alpha + \gamma) \left[ \dot{\alpha} + \dot{\gamma} \right] & -\cos(\alpha + \gamma) \left[ \dot{\alpha} + \dot{\gamma} \right] & -\sin(\alpha)\dot{\alpha} \\
    \cos(\alpha + \gamma) \left[ \dot{\alpha} + \dot{\gamma} \right] & -\sin(\alpha + \gamma) \left[ \dot{\alpha} + \dot{\gamma} \right] & \cos(\alpha)\dot{\alpha} \\
    0 & 0 & 0
\end{bmatrix}
\]
Substituting \(\langle \alpha, \gamma \rangle = \left\langle \frac{\pi}{4}, \frac{\pi}{4} \right\rangle\):
\[
\bm{\mathbf{\dot{G^{s}_{2}}}} = \begin{bmatrix}
    -\left[ \dot{\alpha} + \dot{\gamma} \right] & 0 & -\frac{\sqrt{2}}{2}\dot{\alpha} \\
    0 & -\left[ \dot{\alpha} + \dot{\gamma} \right] & \frac{\sqrt{2}}{2}\dot{\alpha} \\
    0 & 0 & 0
\end{bmatrix}
\]
Given \(\bm{\mathbf{\overline{p}^2}} = \begin{bmatrix}
    x^{'} \\ y^{'} \\ 1
\end{bmatrix}\):
\[
\bm{\mathbf{\dot{\overline{p}}^s}} = \begin{bmatrix}
    -\left[ \dot{\alpha} + \dot{\gamma} \right] & 0 & -\frac{\sqrt{2}}{2}\dot{\alpha} \\
    0 & -\left[ \dot{\alpha} + \dot{\gamma} \right] & \frac{\sqrt{2}}{2}\dot{\alpha} \\
    0 & 0 & 0
\end{bmatrix} \cdot \begin{bmatrix}
    x^{'} \\ y^{'} \\ 1
\end{bmatrix} = \begin{bmatrix}
    -x^{'}\left[ \dot{\alpha} + \dot{\gamma} \right] - \frac{\sqrt{2}}{2}\dot{\alpha} \\
    -y^{'}\left[ \dot{\alpha} + \dot{\gamma} \right] + \frac{\sqrt{2}}{2}\dot{\alpha} \\
    0
\end{bmatrix}
\]
\newpage
\section{}
(b)
\[\]
Outlining the kinematic equations for a two-wheeled robot:
\[
\omega = \frac{r}{T} \cdot (u_r - u_l)
\]
\[
v = \frac{r}{2} \cdot (u_r + u_l)
\]
Solving these equations based on 3(a).i.1:
\[
\omega = 0 \ \frac{rad}{s}, v = 1 \ \frac{m}{s}, T = 0.2 \ m, r = 0.1 \ m
\]
\[
u_r = u_l = 10 \ \frac{rad}{s}
\]
\[\]
Solving these equations based on 3(a).i.2:
\[
\omega = 0.3 \ \frac{rad}{s}, v = 0 \ \frac{m}{s}, T = 0.2 \ m, r = 0.1 \ m
\]
\[
u_r = 0.3 \ \frac{rad}{s}
\]
\[
u_l = -0.3 \ \frac{rad}{s}
\]
Solving these equations based on 3(a).i.3:
\[
\omega = 0.3 \ \frac{rad}{s}, v = 1 \ \frac{m}{s}, T = 0.2 \ m, r = 0.1 \ m
\]
\[
u_r = 10.3 \ \frac{rad}{s}
\]
\[
u_l = 9.7 \ \frac{rad}{s}
\]
\[\]
If the input control parameters were changed from linear and angular velocities to wheel speeds, the state matrix would need to be changed in the implemented solution:
\[
\begin{bmatrix}
    \dot{x} \\ \dot{y} \\ \dot{\theta}
\end{bmatrix} = 
\begin{bmatrix}
    \cos(\theta) & 0 \\ \sin(\theta & 0 \\ 0 & 1
\end{bmatrix} \cdot
\begin{bmatrix}
    v \\ \omega
\end{bmatrix} = \begin{bmatrix}
    \frac{r}{2}\cos(\theta) & \frac{r}{2}\cos(\theta) \\ \frac{r}{2}\sin(\theta) & \frac{r}{2}\sin(\theta) \\ \frac{r}{T} & -\frac{r}{T}
\end{bmatrix} \cdot \begin{bmatrix}
    u_r \\ u_l
\end{bmatrix}
\]
\newpage
\section{}
I.
\[
u_i = \frac{v^{i}_{drive}}{r_i} = \frac{1}{r_i}(v^i_x + v^i_y\tan\gamma_i) = \frac{1}{r_i}\begin{bmatrix}
    1 & \tan\gamma_i
\end{bmatrix} \bm{\mathbf{g_i}}(\theta)\bm{\mathbf{\dot{q}}}
\]
\[
\text{where} \quad \bm{\mathbf{g_i}}(\theta) = \begin{bmatrix}
    \cos(\theta + \beta_i) & \sin(\theta + \beta_i) & x_i\sin(\beta_i) - y_i\cos(\beta_i) \\ -\sin(\theta + \beta_i) & \cos(\theta + \beta_i) & x_i\cos(\beta_i) + y_i\sin(\beta_i)
\end{bmatrix}, \quad \bm{\mathbf{\dot{q}}} = \begin{bmatrix}
    \dot{x} & \dot{y} & \omega
\end{bmatrix}^T
\]
\[\]
For a robot with 3 wheels:

\begin{equation}
    \bm{\mathbf{u}} = \begin{bmatrix}
    u_1 \\ u_2 \\ u_3
\end{bmatrix} = \bm{\mathbf{G}}(\theta)\bm{\mathbf{\dot{q}}}
\end{equation}
\[\]
Since all the Mecanum wheels have the same radius and roller orientation,
\[
r_1 = r_2 = r_3 = r, \quad \gamma_1 = \gamma_2 = \gamma_3 = 0
\]
Finding \(\beta_i, x_i, \text{and } y_i\) for all 3 wheels
\[
\beta_1 = \frac{\pi}{2}, \beta_2 = \frac{7\pi}{6}, \beta_3 = \frac{11\pi}{6}
\]
\[
(x_1, y_1) = (l, 0) \quad (x_2, y_2) = (l\cos(\frac{2\pi}{3}), l\sin(\frac{2\pi}{3})) \quad (x_3, y_3) = (l\cos(\frac{4\pi}{3}), l\sin(\frac{4\pi}{3}))
\]
\[
(x_1, y_1) = (l, 0) \quad (x_2, y_2) = (-\frac{l}{2}, \frac{\sqrt{3}}{2}l) \quad (x_3, y_3) = (-\frac{l}{2}, -\frac{\sqrt{3}}{2}l)
\]
Expanding \(\bm{\mathbf{g_{1,2,3}}}(\theta)\)
\[
\bm{\mathbf{g_1}}(\theta) = \begin{bmatrix}
    -\sin(\theta) & \cos(\theta) & l \\ -\cos(\theta) & -\sin(\theta) & 0
\end{bmatrix}
\]
\[
\renewcommand{\arraystretch}{1.3}
\bm{\mathbf{g_2}}(\theta) = \begin{bmatrix}
    \cos(\theta + \frac{7\pi}{6}) & \sin(\theta + \frac{7\pi}{6}) & -\frac{l}{2}\sin(\frac{7\pi}{6}) - \frac{\sqrt{3}}{2}l\cos(\frac{7\pi}{6}) \\ -\sin(\theta + \frac{7\pi}{6}) & \cos(\theta + \frac{7\pi}{6}) & -\frac{l}{2}\cos(\frac{7\pi}{6}) + \frac{\sqrt{3}}{2}l\sin(\frac{7\pi}{6})
\end{bmatrix}
\]
\[
\renewcommand{\arraystretch}{1.3}
\bm{\mathbf{g_3}}(\theta) = \begin{bmatrix}
    \cos(\theta + \frac{11\pi}{6}) & \sin(\theta + \frac{11\pi}{6}) & -\frac{l}{2}\sin(\frac{11\pi}{6}) + \frac{\sqrt{3}}{2}l\cos(\frac{11\pi}{6}) \\ -\sin(\theta + \frac{11\pi}{6}) & \cos(\theta + \frac{11\pi}{6}) & -\frac{l}{2}\cos(\frac{11\pi}{6}) - \frac{\sqrt{3}}{2}l\sin(\frac{11\pi}{6})
\end{bmatrix}
\]
Evaluating \(\bm{\mathbf{G}}(\theta)\), each row being equivalent to the first row of the minor \(\bm{\mathbf{g_{1,2,3}}}(\theta)\) matrix due to \(\tan(\gamma_{1,2,3}) = 0\)
\[
\renewcommand{\arraystretch}{1.3}
\bm{\mathbf{G}}(\theta) = \frac{1}{r}\begin{bmatrix}
     -\sin(\theta) & \cos(\theta) & l \\  \cos(\theta + \frac{7\pi}{6}) & \sin(\theta + \frac{7\pi}{6}) & -\frac{l}{2}\sin(\frac{7\pi}{6}) - \frac{\sqrt{3}}{2}l\cos(\frac{7\pi}{6}) \\  \cos(\theta + \frac{11\pi}{6}) & \sin(\theta + \frac{11\pi}{6}) & -\frac{l}{2}\sin(\frac{11\pi}{6}) + \frac{\sqrt{3}}{2}l\cos(\frac{11\pi}{6})
\end{bmatrix}
\]
\[\]
\newpage
Full state space equation:
\[
\bm{\mathbf{u}} = \frac{1}{r}\begin{bmatrix}
     -\sin(\theta) & \cos(\theta) & l \\  \cos(\theta + \frac{7\pi}{6}) & \sin(\theta + \frac{7\pi}{6}) & -\frac{l}{2}\sin(\frac{7\pi}{6}) - \frac{\sqrt{3}}{2}l\cos(\frac{7\pi}{6}) \\  \cos(\theta + \frac{11\pi}{6}) & \sin(\theta + \frac{11\pi}{6}) & -\frac{l}{2}\sin(\frac{11\pi}{6}) + \frac{\sqrt{3}}{2}l\cos(\frac{11\pi}{6})
\end{bmatrix} \cdot \begin{bmatrix}
    \dot{x} \\ \dot{y} \\ \omega
\end{bmatrix}
\]
II.
\[\]
1) The \(\bm{\mathbf{\dot{q}}}\) of a straight line with a slope of 60 degrees is:
\[
\renewcommand{\arraystretch}{1.3}
\bm{\mathbf{\dot{q}}} = \begin{bmatrix}
    \frac{1}{2} \\ \frac{\sqrt{3}}{2} \\ 0
\end{bmatrix}
\]
Using an initial state of \(x_0 = 0 [cm], \ y_0 = 0 [cm], \ \theta_0 = 0 [rad], \ l = 25 [cm], \ r = 10 [cm]\) and solving for \(\bm{\mathbf{u}}\)
\[
u_1 = u_2, \quad u_3 = 0
\]
2) The \(\bm{\mathbf{\dot{q}}}\) of a circular path with a diameter of 2 m = 200 cm is:
\[
\renewcommand{\arraystretch}{1.3}
\bm{\mathbf{\dot{q}}} = \begin{bmatrix}
    0 \\ \frac{d\pi}{T} \\ \frac{2\pi}{T}
\end{bmatrix}
\]
where \(d = 200 \ cm, \quad T = total \ simulation \ time\)
\[\]
Using an initial state of \(x_0 = 0 [cm], \ y_0 = 0 [cm], \ \theta_0 = 0 [rad], \ l = 25 [cm], \ r = 10 [cm]\) and solving for \(\bm{\mathbf{u}}\)

\[
u_2 = u_3, \quad u_1 = -5u_2
\]
\end{document}
