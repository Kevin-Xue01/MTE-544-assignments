\section{RANSAC with Ellipse I (18)}

In this exercise, you are tasked to detect an ellipse based on given dataset using RANSAC.
\begin{figure}[H]
    \centering
    \includegraphics[width=0.65\textwidth]{img/ellipse.png}
    \caption{A Plot of Randomly Selected Dataset}
    \label{fig:p3c1}
\end{figure}
The goal is to estimate the parameters $\mathbf{Q}$ and $\mathbf{b}$ by fitting the sensor data to an ellipse. 
An ellipse can be described by the quadratic equation: 
\begin{equation*}
(\mathbf{p}-\mathbf{b})^\top\mathbf{Q}(\mathbf{p}-\mathbf{b})=1
\end{equation*}
or
\begin{equation*}
    Ax^2+2Bxy+Cy^2-2Dx-2Ey+1=0
\end{equation*}
where:
\begin{itemize}
\item $\mathbf{Q}=\mathbf{Q}^\top\in\mathbb{R}^{2\times2}$ is a unique symmetric positive definite matrix of coefficients. [If $\mathbf{Q}$ is not positive definite, you may have a hyperbola instead.]
\end{itemize}
Note that the coefficients ($A$,$B$,$C$,$D$,and $E$) are also unique.
The conversion from the polynomial coeffcients to the quadratic form can be expressed as follows:
\begin{equation*}
    \begin{aligned}
        \mathbf{Q}&=\alpha\begin{bmatrix}
        A & B\\ B & C
    \end{bmatrix}\\
    \mathbf{Q}\mathbf{b}&=\alpha\begin{bmatrix}
        D\\E
    \end{bmatrix}\\
    \alpha&=\frac{1}{\begin{bmatrix}
        D & E
    \end{bmatrix}\begin{bmatrix}
        A & B \\ B & C
    \end{bmatrix}^{-1}\begin{bmatrix}
        D \\ E
    \end{bmatrix}-1}
    \end{aligned}
\end{equation*}

\begin{enumerate}[a.)]  
\item Determine the minimum number of points to uniquely determined ellipse's parameters as well as derive a method to construct an ellipse with the parameters ($A$,$B$,$C$,$D$, and $E$) based on those points. $\mathbf{p}_i=(x_i,y_i)$ denote the $i^\text{th}$ data point. You must explain your rationale to receive credits. [10 pts]
\[\]
Answer:
The general form of an ellipse is:
\begin{equation*}
    Ax^2+2Bxy+Cy^2-2Dx-2Ey+1=0
\end{equation*}
From this equation, it is evident that with five unknowns, five points are needed to uniquely determine an ellipse's parameters.
Each point is written as:
\[
p_i = (x_i, y_i)
\]
Substituting \(x = x_i\) and \(y = y_i\) for \(i = 1, 2, 3, 4, 5\) into the ellipse equation.
This results in a system of five linear equations:
\[
\begin{cases}
A x_1^2 + B x_1 y_1 + C y_1^2 + D x_1 + E y_1 = 1 \\
A x_2^2 + B x_2 y_2 + C y_2^2 + D x_2 + E y_2 = 1 \\
A x_3^2 + B x_3 y_3 + C y_3^2 + D x_3 + E y_3 = 1 \\
A x_4^2 + B x_4 y_4 + C y_4^2 + D x_4 + E y_4 = 1 \\
A x_5^2 + B x_5 y_5 + C y_5^2 + D x_5 + E y_5 = 1 \\
\end{cases}
\]
Rewriting the above system of linear equations:
\[
\begin{bmatrix}
x_1^2 & x_1 y_1 & y_1^2 & x_1 & y_1 \\
x_2^2 & x_2 y_2 & y_2^2 & x_2 & y_2 \\
x_3^2 & x_3 y_3 & y_3^2 & x_3 & y_3 \\
x_4^2 & x_4 y_4 & y_4^2 & x_4 & y_4 \\
x_5^2 & x_5 y_5 & y_5^2 & x_5 & y_5 \\
\end{bmatrix}
\begin{bmatrix}
A \\ B \\ C \\ D \\ E
\end{bmatrix}
=
\begin{bmatrix}
1 \\ 1 \\ 1 \\ 1 \\ 1
\end{bmatrix}
\]
Solving:
\[
\begin{bmatrix}
A \\ B \\ C \\ D \\ E
\end{bmatrix}
=
\begin{bmatrix}
x_1^2 & x_1 y_1 & y_1^2 & x_1 & y_1 \\
x_2^2 & x_2 y_2 & y_2^2 & x_2 & y_2 \\
x_3^2 & x_3 y_3 & y_3^2 & x_3 & y_3 \\
x_4^2 & x_4 y_4 & y_4^2 & x_4 & y_4 \\
x_5^2 & x_5 y_5 & y_5^2 & x_5 & y_5 \\
\end{bmatrix}^{-1}
\begin{bmatrix}
1 \\ 1 \\ 1 \\ 1 \\ 1
\end{bmatrix}
\]
\item 
Fit an ellipse with the following datasets and compute for $\mathbf{Q}$, $\mathbf{b}$, $A$, $B$, $C$, $D$ and $E$.

\begin{equation*}
    (2.92, -6.01), \quad (3.40, -7.20), \quad (4.99, -7.84), \quad (5.48, -7.04), \quad (4.20, -5.91)
\end{equation*}

 You don't have code for this part, but you still have to do it with Python in the next part. [8 pts.]
\end{enumerate}
